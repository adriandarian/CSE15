\documentclass[a4paper]{article}

  \usepackage{fullpage} % Package to use full page
  \usepackage{parskip} % Package to tweak paragraph skipping
  \usepackage{tikz} % Package for drawing
  \usepackage{amsmath}
  \usepackage{amsfonts}
  \usepackage{amssymb}
  \usepackage{hyperref}
  \usepackage[utf8]{inputenc}
  \usepackage[english]{babel}
  \usepackage{multicol}
  
  \newcommand\tab[1][0.5cm]{\hspace*{#1}}
  
  \title{Discrete Mathematics: HW2}
  \author{Adrian Darian}
  \date{2018/9/14}
  
  \begin{document}
  
  \maketitle
  
  \section*{1.4 Predicates and Quantifiers}
  \begin{itemize}
    \item[6] Let $N(x)$ be the statement "$x$ has visited North Dakota," where the domain consists of the students in your school. Express each of these quantifications in English. \\
      d. $\exists x \neg N(x)$ \\
      e. $\neg \forall x N(x)$ \\
      f. $\forall x \neg N(x)$
    \item[8] Translate these statements into English, where $R(x)$ is "$x$ is a rabbit" and $H(x)$ is "$x$ hops" and the domain consists of all animals. \\
      c. $\exists x (R(x) \rightarrow H(x))$ \\
      d. $\exists x (R(x) \land H(x))$
    \item[10] Let $C(x)$ be the statement "$x$ has a cat," let $D(x)$ be the statement "$x$ has a dog," and let $F(x)$ be the statement "$x$ has a ferret." Express each of these statements in terms of $C(x), D(x), F(x),$ quantifiers, and logical connectives. Let the domain consist of all students in your class. \\
      a. A student in your class has a cat, a dog, and a ferret. \\
      c. Some student in your class has a cat and a ferret, but not a dog. \\
      e. For each of the three animals, cats, dogs, and ferrets, there is a student in your class who has this animal as a pet.
    \item[14] Determine the truth value of each of these statements if the domain consists of all real numbers. \\
      a. $\exists x(x^3 = -1)$ \\
      b. $\exists x(x^4 < x^2)$ 
    \item[24] \\
      c. \\
      d.
    \item[28] \\
      a. \\
      b.
    \item[34] \\
      c. \\
      d.
    \item[40] \\
      a. \\
      b.
    \item[42] \\
      c. \\
      d.
  \end{itemize}

  
  \section*{1.5}
  \begin{itemize}
    \item[4] \\
      b. \\
      c. \\
      d.
    \item[8] \\
      a. \\
      b. 
    \item[12] \\
      f. \\
      g. \\
      i.
    \item[18] \\
      a. \\
      c. 
    \item[24] \\
      a. \\
      b.
    \item[28] \\
      c. \\
      d. \\
      e.
    \item[30] \\
      a. \\
      b. \\
      c.
    \item[36] \\
      a. \\
      b. 
  \end{itemize}

  
  \end{document}